\documentclass[12pt]{jsarticle}
\usepackage{geometry}
\geometry{left=5mm,right=5mm,top=5mm,bottom=5mm}		
\usepackage{amssymb}
\usepackage{mathcomp}
\usepackage{amsmath,amsthm} 
\usepackage[dvipdfmx]{graphicx}
\usepackage{txfonts}
\newenvironment{solution}
  {\renewcommand\qedsymbol{$\blacksquare$}\begin{proof}[Solution]}
  {\end{proof}}
\begin{titlepage}
\title{「船型試験水槽を用いた円柱のカルマン渦の可視化」}
\author{工学部航空宇宙工学科三年 野本陽平 (03-180332)}
\date{2019年1月8日}
\thispagestyle{empty}
\end{titlepage}
\begin{document}
\maketitle
\section{船形試験水槽について}
船形試験水槽は, 長さ85m, 幅3.5m, 深さ2.4m, 最大速度は3.0m/s(本実験での最大速度は0.7m/s)で1937年に建設された. 実験は水槽の淵のレール上を台車が走り, 台車に取り付けられた模型を台車に乗った人が観測することで行う. 台車の下方についた 3 分力計を用いて抵抗等を測定するとみられる. 風洞実験と異なる点としては, 観測者は慣性系の視点から観察できる点, 水面に波が現れることから可視化のために煙やタフト, 光学的なツール等を用いる必要がない点等が挙げられる.
\section{カルマン渦の可視化実験について}
直径40cmの円柱を深さ0mm(水面にギリギリ接している状態), 200mm, 500mmに沈め, 速度は0から最大の0.7m/sまでステップ状に0.1m/sずつ加速していく形で行なった. 水中を移動する物体の抵抗を支配する無次元パラメータを以下の通り計算した. ただし速度は最大速度のものを, 水温は$5C^{\circ}$として動粘性係数を引用し, 代表長さには円柱の直径を用いた.
\[Re = \cfrac{\rho VL}{\mu} = \cfrac{VL}{\nu} =\cfrac{0.7 \cdot 0.4}{1.52 \times 10^{-6}} \approx 1.84 \times 10^5 , \indent Fr = \cfrac{V}{\sqrt{Lg}} = \cfrac{0.7}{\sqrt{0.4 \cdot 9.8}} \approx 0.354 \]
円柱前方では, 円柱の深さに関係なく速度が速くなればなるほど波面の間隔が密になり, 円柱周りの水位が高くなる様子が観察された. 円柱後方では, 200mm, 500mmの時は0.3m/s程度からカルマン渦が観測され始めたのに対し, 0mmの時は最後まで観測できなかった. これは円柱下部から後方への水量が多く, カルマン渦の形成を邪魔してしまったからだと考えられる. 図\ref{a}にカルマン渦の様子を載せている.
\section{航空宇宙工学分野への応用について}
水だと重力の影響を受けすぎるので, 水槽から水を抜き, 煙を充満させた上で同種の実験を行うのはどうだろうか. 前方と後方に風洞の吹き込み口と吹き出し口を設置し, 相対速度は合わせた上で台車が移動している時とそうでない時の差を比べてみたい(今更言うことでもないかもしれないが, 相対速度的な概念は自分は完全には納得していない). 実際に慣性系に立って見ることができる風洞は珍しいはずなので, それだけでも価値があるかもしれない.
\section{感想}
円柱の後方に広がる渦を見ていると, 久しぶりに水の流れを見た気がした. 普段意識して水面の波紋を見つめることはないので, そこに日々学習している渦があるなんて気付いていなかっただけなのかもしれない. 水面から聞こえる音を聞いていると, 数値計算がどれほど使われる世の中になったとしても, 情報量的な意味で実際に使われる物に携わる工学では実物を見て感じることが大事なのだと思った. というより, 数値計算は金銭的・時間的予算が不足している際に使われる簡易的手段に過ぎないのだな, 古風と言ってしまえばそれまでだが, こういう実際に行う設備は贅沢なものなのだなと改めて感じた.
\begin{figure}[htbp]
\begin{center}
\includegraphics[width=10cm]{a.png}
\caption{カルマン渦の様子(深さ500mm, 速度0.7m/s)}
\label{a}
\end{center}
\end{figure}
\end{document}